% ============================================
% CHAPTER 1: KONSEP DAN PERAN K3L
% Bobot: CPMK5-5 = 25
% ============================================

% JUDUL KASUS - SILAKAN ISI
\subsection*{Judul Kasus: \underline{\hspace{10cm}}}
\addcontentsline{toc}{subsection}{Judul Kasus}

\vspace{0.5cm}

% ========== BAGIAN A: DESKRIPSI SITUASI ==========
\subsection{Deskripsi Situasi}
\label{subsec:deskripsi-situasi}

% Petunjuk:
% - Jelaskan situasi atau kasus K3L yang dipilih secara detail
% - Sertakan data atau pengamatan dari lingkungan sekitar
% - Gunakan bahasa formal dan sistematis
% - Minimal 2-3 paragraf

Jelaskan situasi atau kondisi yang menjadi fokus analisis K3L. Deskripsikan dengan detail meliputi:
\begin{itemize}
    \item Lokasi dan jenis industri/tempat kerja yang diamati
    \item Kondisi aktual yang ditemukan di lapangan
    \item Permasalahan atau potensi bahaya yang teridentifikasi
    \item Data kuantitatif atau kualitatif pendukung
\end{itemize}

\textit{[Tulis deskripsi situasi Anda di sini dengan bahasa formal dan sistematis. Sertakan data observasi dari lingkungan sekitar.]}

\vspace{1cm}

% ========== BAGIAN B: ANALISIS KONSEP K3L ==========
\subsection{Analisis Konsep K3L}
\label{subsec:analisis-konsep}

% Petunjuk:
% - Kaitkan situasi dengan konsep-konsep K3L yang relevan
% - Jelaskan teori/prinsip K3L yang berlaku
% - Analisis bagaimana konsep K3L diterapkan atau diabaikan

Analisis situasi berdasarkan konsep-konsep K3L yang meliputi:

\subsubsection{Keamanan (Security)}
\textit{[Analisis aspek keamanan dalam kasus yang dipilih]}

\subsubsection{Kesehatan Kerja (Occupational Health)}
\textit{[Analisis aspek kesehatan kerja dan dampaknya terhadap pekerja]}

\subsubsection{Keselamatan Kerja (Occupational Safety)}
\textit{[Analisis aspek keselamatan kerja dan potensi kecelakaan]}

\subsubsection{Lingkungan Kerja (Work Environment)}
\textit{[Analisis kondisi lingkungan kerja dan dampak lingkungan]}

\vspace{1cm}

% ========== BAGIAN C: PERAN K3L DI INDUSTRI ==========
\subsection{Peran K3L di Industri}
\label{subsec:peran-k3l}

% Petunjuk:
% - Jelaskan pentingnya K3L dalam konteks industri yang dibahas
% - Diskusikan manfaat implementasi K3L
% - Berikan contoh konkret peran K3L

Uraikan peran dan pentingnya K3L dalam industri/tempat kerja yang dianalisis, meliputi:

\begin{enumerate}[label=\alph*.]
    \item \textbf{Peran Preventif}
    
    \textit{[Jelaskan bagaimana K3L berperan dalam pencegahan kecelakaan dan penyakit akibat kerja]}
    
    \item \textbf{Peran Protektif}
    
    \textit{[Jelaskan bagaimana K3L melindungi pekerja, aset, dan lingkungan]}
    
    \item \textbf{Peran Produktif}
    
    \textit{[Jelaskan hubungan K3L dengan produktivitas dan efisiensi kerja]}
    
    \item \textbf{Peran Legal dan Etis}
    
    \textit{[Jelaskan aspek hukum dan etika dalam implementasi K3L]}
\end{enumerate}

\vspace{1cm}

% ========== BAGIAN D: STRATEGI KOMUNIKASI ==========
\subsection{Strategi Komunikasi}
\label{subsec:strategi-komunikasi}

% Petunjuk:
% - Rancang strategi komunikasi K3L yang efektif
% - Jelaskan metode penyampaian informasi K3L
% - Identifikasi stakeholder dan cara berkomunikasi dengan mereka

Strategi komunikasi K3L yang direkomendasikan untuk kasus ini:

\subsubsection{Identifikasi Stakeholder}
\textit{[Identifikasi pihak-pihak yang terlibat: manajemen, pekerja, kontraktor, dll.]}

\subsubsection{Metode Komunikasi}
\textit{[Jelaskan metode komunikasi yang akan digunakan: briefing, poster, training, dll.]}

Contoh metode komunikasi:
\begin{itemize}
    \item Safety briefing/toolbox meeting
    \item Signage dan rambu-rambu K3L
    \item Sosialisasi dan pelatihan
    \item Media komunikasi internal (bulletin, email, dll.)
    \item Sistem pelaporan bahaya
\end{itemize}

\subsubsection{Media dan Saluran Komunikasi}
\textit{[Tentukan media dan saluran yang paling efektif untuk menyampaikan informasi K3L]}

\subsubsection{Evaluasi Efektivitas Komunikasi}
\textit{[Jelaskan bagaimana mengukur efektivitas strategi komunikasi yang dirancang]}

% ========== CATATAN ==========
% Pastikan semua bagian ditulis dengan:
% - Bahasa formal dan ilmiah
% - Sistematis dan terstruktur
% - Didukung data/referensi yang relevan
% - Minimal rujuk 2-3 sumber literatur