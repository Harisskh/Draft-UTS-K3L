% ============================================
% CHAPTER 2: REGULASI DAN STANDAR K3L
% Bobot: CPMK5-5 = 25
% ============================================

% JUDUL KASUS - SILAKAN ISI
\subsection*{Judul Kasus: \underline{\hspace{10cm}}}
\addcontentsline{toc}{subsection}{Judul Kasus}

\vspace{0.5cm}

% ========== BAGIAN A: ANALISIS KETERKAITAN REGULASI ==========
\subsection{Analisis Keterkaitan Regulasi}
\label{subsec:analisis-regulasi}

% Petunjuk:
% - Identifikasi regulasi K3L yang relevan dengan kasus
% - Jelaskan bagaimana regulasi tersebut berkaitan dengan situasi
% - Analisis kepatuhan atau ketidakpatuhan terhadap regulasi

Analisis regulasi dan standar K3L yang berkaitan dengan kasus yang dipilih:

\subsubsection{Regulasi Nasional}

\textbf{Undang-Undang yang Relevan:}
\begin{itemize}
    \item UU No. 1 Tahun 1970 tentang Keselamatan Kerja
    \item UU No. 13 Tahun 2003 tentang Ketenagakerjaan
    \item UU No. 32 Tahun 2009 tentang Perlindungan dan Pengelolaan Lingkungan Hidup
    \item \textit{[Tambahkan UU lain yang relevan]}
\end{itemize}

\textit{[Jelaskan pasal-pasal spesifik yang terkait dengan kasus Anda]}

\vspace{0.5cm}

\textbf{Peraturan Pemerintah dan Peraturan Menteri:}
\begin{itemize}
    \item PP No. 50 Tahun 2012 tentang Penerapan SMK3
    \item Permenaker No. 5 Tahun 2018 tentang K3 Lingkungan Kerja
    \item Permenaker No. 8 Tahun 2020 tentang Alat Pelindung Diri
    \item \textit{[Tambahkan peraturan lain yang relevan]}
\end{itemize}

\textit{[Analisis bagaimana peraturan-peraturan ini berkaitan dengan kasus]}

\vspace{0.5cm}

\subsubsection{Standar Internasional}

\textbf{Standar ISO dan OHSAS:}
\begin{itemize}
    \item ISO 45001:2018 - Sistem Manajemen Kesehatan dan Keselamatan Kerja
    \item ISO 14001:2015 - Sistem Manajemen Lingkungan
    \item \textit{[Tambahkan standar lain yang relevan]}
\end{itemize}

\textit{[Jelaskan penerapan atau ketidaksesuaian dengan standar internasional]}

\vspace{0.5cm}

\subsubsection{Standar Industri Spesifik}

\textit{[Identifikasi standar khusus untuk industri yang dibahas, misalnya: standar kelistrikan, standar konstruksi, standar laboratorium, dll.]}

\vspace{0.5cm}

\subsubsection{Analisis Kepatuhan}

Berdasarkan observasi dan data yang dikumpulkan, analisis kepatuhan terhadap regulasi:

\begin{table}[h]
\centering
\begin{tabular}{|p{5cm}|p{3cm}|p{6cm}|}
\hline
\textbf{Aspek Regulasi} & \textbf{Status Kepatuhan} & \textbf{Keterangan} \\
\hline
\textit{[Contoh: Penyediaan APD]} & \textit{[Patuh/Tidak]} & \textit{[Penjelasan detail]} \\
\hline
 & & \\
\hline
 & & \\
\hline
 & & \\
\hline
\end{tabular}
\caption{Analisis Kepatuhan Regulasi K3L}
\label{tab:kepatuhan-regulasi}
\end{table}

\textit{[Tambahkan penjelasan narasi tentang temuan kepatuhan]}

\vspace{1cm}

% ========== BAGIAN B: LANGKAH PERBAIKAN SISTEM ==========
\subsection{Langkah Perbaikan Sistem}
\label{subsec:perbaikan-sistem}

% Petunjuk:
% - Rancang langkah-langkah konkret untuk memperbaiki sistem K3L
% - Prioritaskan perbaikan berdasarkan tingkat urgensi dan risiko
% - Sertakan timeline dan resources yang dibutuhkan

Berdasarkan analisis regulasi dan kepatuhan, berikut langkah perbaikan sistem K3L:

\subsubsection{Perbaikan Jangka Pendek (0-3 bulan)}

\begin{enumerate}
    \item \textbf{Perbaikan Segera (Emergency)}
    
    \textit{[Identifikasi masalah kritis yang memerlukan tindakan segera]}
    
    \begin{itemize}
        \item Masalah: \textit{[Deskripsikan]}
        \item Tindakan: \textit{[Deskripsikan]}
        \item Penanggung jawab: \textit{[Sebutkan]}
        \item Target waktu: \textit{[Tentukan]}
    \end{itemize}
    
    \item \textbf{Perbaikan Prioritas Tinggi}
    
    \textit{[Masalah yang perlu diperbaiki dalam waktu dekat]}
\end{enumerate}

\vspace{0.5cm}

\subsubsection{Perbaikan Jangka Menengah (3-6 bulan)}

\textit{[Uraikan program perbaikan yang memerlukan perencanaan lebih matang]}

\begin{itemize}
    \item Pengembangan sistem manajemen K3L
    \item Upgrading fasilitas dan peralatan
    \item Implementasi program kesehatan kerja
    \item \textit{[Tambahkan poin lainnya]}
\end{itemize}

\vspace{0.5cm}

\subsubsection{Perbaikan Jangka Panjang (6-12 bulan)}

\textit{[Uraikan program strategis dan berkelanjutan]}

\begin{itemize}
    \item Sertifikasi ISO 45001
    \item Budaya keselamatan (safety culture)
    \item Sistem monitoring dan evaluasi berkelanjutan
    \item \textit{[Tambahkan poin lainnya]}
\end{itemize}

\vspace{0.5cm}

\subsubsection{Rencana Implementasi}

\begin{table}[h]
\centering
\small
\begin{tabular}{|c|p{4cm}|p{3cm}|c|p{2.5cm}|}
\hline
\textbf{No} & \textbf{Langkah Perbaikan} & \textbf{Penanggung Jawab} & \textbf{Timeline} & \textbf{Resources} \\
\hline
1 & \textit{[Langkah 1]} & \textit{[PJ]} & \textit{[Waktu]} & \textit{[Sumber daya]} \\
\hline
2 & & & & \\
\hline
3 & & & & \\
\hline
4 & & & & \\
\hline
5 & & & & \\
\hline
\end{tabular}
\caption{Rencana Implementasi Perbaikan Sistem K3L}
\label{tab:rencana-implementasi}
\end{table}

\vspace{1cm}

% ========== BAGIAN C: PERAN KOMUNIKASI & PELATIHAN ==========
\subsection{Peran Komunikasi dan Pelatihan}
\label{subsec:komunikasi-pelatihan}

% Petunjuk:
% - Jelaskan pentingnya komunikasi dalam implementasi regulasi
% - Rancang program pelatihan K3L
% - Identifikasi metode komunikasi yang efektif

\subsubsection{Peran Komunikasi dalam Implementasi Regulasi}

\textit{[Jelaskan bagaimana komunikasi efektif mendukung penerapan regulasi K3L]}

Aspek-aspek komunikasi yang perlu diperhatikan:
\begin{itemize}
    \item Sosialisasi peraturan dan kebijakan K3L
    \item Penyampaian perubahan regulasi terbaru
    \item Sistem pelaporan dan feedback
    \item Komunikasi antar level organisasi
\end{itemize}

\vspace{0.5cm}

\subsubsection{Program Pelatihan K3L}

\textbf{A. Analisis Kebutuhan Pelatihan}

\textit{[Identifikasi gap kompetensi dan kebutuhan pelatihan]}

\vspace{0.3cm}

\textbf{B. Rancangan Program Pelatihan}

\begin{table}[h]
\centering
\small
\begin{tabular}{|c|p{4cm}|p{3cm}|p{2cm}|p{2cm}|}
\hline
\textbf{No} & \textbf{Jenis Pelatihan} & \textbf{Target Peserta} & \textbf{Durasi} & \textbf{Frekuensi} \\
\hline
1 & Induksi K3L & Karyawan baru & 4 jam & Sekali \\
\hline
2 & K3L Umum & Semua karyawan & 8 jam & Tahunan \\
\hline
3 & \textit{[Pelatihan spesifik]} & \textit{[Target]} & \textit{[Durasi]} & \textit{[Frekuensi]} \\
\hline
4 & & & & \\
\hline
\end{tabular}
\caption{Program Pelatihan K3L}
\label{tab:program-pelatihan}
\end{table}

\vspace{0.3cm}

\textbf{C. Metode Pelatihan}

\begin{itemize}
    \item Classroom training
    \item On-the-job training
    \item E-learning
    \item Simulasi dan drill
    \item \textit{[Tambahkan metode lainnya]}
\end{itemize}

\vspace{0.5cm}

\subsubsection{Evaluasi Efektivitas Komunikasi dan Pelatihan}

\textit{[Jelaskan metode evaluasi untuk mengukur efektivitas program komunikasi dan pelatihan]}

\begin{itemize}
    \item Pre-test dan post-test
    \item Observasi perilaku kerja
    \item Survey kepuasan peserta
    \item Monitoring indikator K3L (frequency rate, severity rate, dll.)
    \item \textit{[Tambahkan metode evaluasi lainnya]}
\end{itemize}

% ========== CATATAN ==========
% Pastikan semua bagian ditulis dengan:
% - Merujuk pada regulasi yang spesifik dan terbaru
% - Data dan fakta yang dapat diverifikasi
% - Rencana yang realistis dan dapat diimplementasikan
% - Bahasa formal dan sistematis