% ============================================
% CHAPTER 4: PENERAPAN SOP & SIKAP AKADEMIK
% Bobot: CPMK5-6 = 25
% ============================================

% ========== BAGIAN A: RANCANGAN SINGKAT SOP ==========
\subsection{Rancangan Singkat SOP}
\label{subsec:rancangan-sop}

% Petunjuk:
% - Buat SOP yang spesifik untuk kasus yang dipilih
% - Gunakan bahasa yang jelas, sistematis, dan operasional
% - Sertakan langkah-langkah detail dengan penanggung jawab

\subsubsection{Informasi Umum SOP}

\begin{table}[h]
\centering
\begin{tabular}{|p{4cm}|p{9cm}|}
\hline
\textbf{Judul SOP} & \textit{[Isi judul SOP sesuai kasus, contoh: "SOP Pengoperasian Mesin Bubut"]} \\
\hline
\textbf{Nomor Dokumen} & \textit{[Contoh: SOP/K3L/001/2025]} \\
\hline
\textbf{Tanggal Berlaku} & \textit{[Tanggal efektif]} \\
\hline
\textbf{Revisi} & \textit{[Nomor revisi, contoh: Rev. 00]} \\
\hline
\textbf{Disusun Oleh} & \textit{[Nama penyusun]} \\
\hline
\textbf{Disetujui Oleh} & \textit{[Nama yang menyetujui]} \\
\hline
\textbf{Unit Kerja} & \textit{[Nama unit/departemen]} \\
\hline
\end{tabular}
\caption{Informasi Dokumen SOP}
\label{tab:info-sop}
\end{table}

\vspace{0.5cm}

\subsubsection{Tujuan SOP}

\textit{[Jelaskan tujuan dari SOP ini, misalnya:]}
\begin{itemize}
    \item Memastikan keselamatan pekerja dalam melakukan aktivitas
    \item Meminimalkan risiko kecelakaan kerja
    \item Menjamin konsistensi prosedur kerja
    \item Memenuhi persyaratan regulasi K3L
\end{itemize}

\vspace{0.5cm}

\subsubsection{Ruang Lingkup}

\textit{[Jelaskan cakupan penerapan SOP, misalnya:]}
\begin{itemize}
    \item Area kerja yang tercakup
    \item Personel yang terlibat
    \item Aktivitas yang diatur
    \item Batasan-batasan penerapan
\end{itemize}

\vspace{0.5cm}

\subsubsection{Definisi dan Istilah}

\textit{[Jelaskan istilah-istilah khusus yang digunakan dalam SOP]}

\begin{itemize}
    \item \textbf{APD}: Alat Pelindung Diri
    \item \textbf{JSA}: Job Safety Analysis
    \item \textbf{LOTO}: Lock Out Tag Out
    \item \textit{[Tambahkan definisi lain yang relevan]}
\end{itemize}

\vspace{0.5cm}

\subsubsection{Referensi}

\textit{[Sebutkan dokumen atau regulasi yang menjadi acuan]}

\begin{itemize}
    \item UU No. 1 Tahun 1970 tentang Keselamatan Kerja
    \item ISO 45001:2018
    \item \textit{[Tambahkan referensi lain]}
\end{itemize}

\vspace{1cm}

\subsubsection{Prosedur Kerja}

\clearpage
\begin{landscape}
    \thispagestyle{empty} % <<< INI PERINTAHNYA: Menghilangkan semua header & footer
    
    \begin{longtable}{|c|p{5cm}|p{3cm}|p{4cm}|p{4cm}|}
        \caption{Langkah-Langkah SOP K3L} \label{tab:langkah-sop} \\
        \hline
        \textbf{Langkah} & \textbf{Deskripsi Tindakan} & \textbf{Penanggung Jawab} & \textbf{Alat yang Digunakan} & \textbf{Catatan K3L} \\
        \hline
        \endfirsthead
        
        \multicolumn{5}{c}%
        {\tablename\ \thetable\ -- \textit{Lanjutan dari halaman sebelumnya}} \\
        \hline
        \textbf{Langkah} & \textbf{Deskripsi Tindakan} & \textbf{Penanggung Jawab} & \textbf{Alat yang Digunakan} & \textbf{Catatan K3L} \\
        \hline
        \endhead

        \hline
        \multicolumn{5}{r}{\textit{Bersambung ke halaman berikutnya}} \\
        \endfoot

        \hline
        \endlastfoot

        % ... (semua baris tabel Anda) ...
        \hline
    \end{longtable}
    
\end{landscape}
\clearpage  

\vspace{0.5cm}

\subsubsection{APD yang Diperlukan}

\begin{table}[h]
\centering
\begin{tabular}{|c|p{4cm}|p{6cm}|p{3cm}|}
\hline
\textbf{No} & \textbf{Jenis APD} & \textbf{Fungsi} & \textbf{Standar} \\
\hline
1 & Safety Helmet & Melindungi kepala dari benturan & SNI, ANSI \\
\hline
2 & Safety Shoes & Melindungi kaki dari tertimpa benda & SNI \\
\hline
3 & Safety Glasses & Melindungi mata dari percikan & ANSI Z87.1 \\
\hline
4 & \textit{[Tambahkan APD lain]} & & \\
\hline
\end{tabular}
\caption{Daftar APD yang Diperlukan}
\label{tab:apd-sop}
\end{table}

\vspace{0.5cm}

\subsubsection{Kondisi Darurat}

\textit{[Jelaskan prosedur jika terjadi kondisi darurat]}

\begin{enumerate}
    \item \textbf{Stop Work}: Hentikan pekerjaan segera
    \item \textbf{Secure Area}: Amankan area kerja
    \item \textbf{Report}: Laporkan ke supervisor/HSE
    \item \textbf{Emergency Response}: Lakukan tindakan darurat sesuai jenis insiden
    \item \textbf{Investigation}: Lakukan investigasi kecelakaan
\end{enumerate}

\vspace{0.5cm}

\subsubsection{Review dan Update SOP}

\textit{[Jelaskan mekanisme review dan update SOP]}

\begin{itemize}
    \item Review berkala: minimal 1 tahun sekali
    \item Review insidental: jika terjadi perubahan proses atau insiden
    \item Approval: SOP harus disetujui oleh manajemen
    \item Sosialisasi: SOP baru harus disosialisasikan ke semua pihak terkait
\end{itemize}

\vspace{1cm}

% ========== BAGIAN B: SIKAP AKADEMIK DAN KOMUNIKATIF ==========
\subsection{Sikap Akademik dan Komunikatif}
\label{subsec:sikap-akademik}

% Petunjuk:
% - Refleksikan proses pembelajaran dalam mengerjakan UTS
% - Tunjukkan sikap aktif, analitis, inovatif, dan bertanggung jawab
% - Evaluasi kemampuan komunikasi dalam menyusun laporan

\subsubsection{Refleksi Proses Pembelajaran}

\textit{[Tuliskan refleksi pribadi tentang proses pembelajaran K3L]}

\textbf{A. Pemahaman Konsep K3L}

\textit{[Jelaskan bagaimana pemahaman Anda tentang K3L berkembang selama perkuliahan dan dalam mengerjakan UTS ini]}

Aspek-aspek yang dipelajari:
\begin{itemize}
    \item Konsep dasar K3L dan penerapannya
    \item Regulasi dan standar yang berlaku
    \item Metode identifikasi bahaya dan analisis risiko
    \item Penyusunan SOP yang efektif
    \item \textit{[Tambahkan aspek pembelajaran lainnya]}
\end{itemize}

\vspace{0.5cm}

\textbf{B. Tantangan yang Dihadapi}

\textit{[Jelaskan tantangan atau kesulitan dalam mengerjakan UTS ini dan bagaimana mengatasinya]}

\begin{enumerate}
    \item Tantangan: \textit{[Deskripsikan]}
    
    Solusi: \textit{[Jelaskan cara mengatasi]}
    
    \item Tantangan: \textit{[Deskripsikan]}
    
    Solusi: \textit{[Jelaskan cara mengatasi]}
\end{enumerate}

\vspace{0.5cm}

\subsubsection{Sikap Aktif dalam Pembelajaran}

\textit{[Tunjukkan bagaimana Anda aktif dalam proses pembelajaran]}

\textbf{A. Inisiatif Belajar}

\begin{itemize}
    \item Melakukan observasi langsung di \textit{[sebutkan lokasi]}
    \item Mencari referensi tambahan dari \textit{[sebutkan sumber]}
    \item Berdiskusi dengan \textit{[sebutkan pihak terkait]}
    \item \textit{[Tambahkan inisiatif lainnya]}
\end{itemize}

\vspace{0.5cm}

\textbf{B. Partisipasi dan Kolaborasi}

\textit{[Jelaskan jika ada kolaborasi dengan pihak lain dalam pengumpulan data atau analisis]}

\vspace{0.5cm}

\subsubsection{Sikap Analitis dan Inovatif}

\textit{[Tunjukkan kemampuan berpikir analitis dan inovatif]}

\textbf{A. Kemampuan Analisis}

\textit{[Jelaskan bagaimana Anda menganalisis kasus K3L secara kritis dan mendalam]}

Pendekatan analisis yang digunakan:
\begin{itemize}
    \item Analisis data observasi
    \item Komparasi dengan standar/best practice
    \item Root cause analysis
    \item \textit{[Tambahkan pendekatan lainnya]}
\end{itemize}

\vspace{0.5cm}

\textbf{B. Inovasi dan Kreativitas}

\textit{[Jelaskan ide-ide inovatif yang Anda tawarkan dalam mengatasi permasalahan K3L]}

Contoh inovasi:
\begin{itemize}
    \item \textit{[Ide inovatif 1]}
    \item \textit{[Ide inovatif 2]}
    \item \textit{[Ide inovatif 3]}
\end{itemize}

\vspace{0.5cm}

\subsubsection{Kemampuan Komunikasi}

\textit{[Evaluasi kemampuan komunikasi dalam menyusun laporan ini]}

\textbf{A. Komunikasi Tertulis}

Aspek komunikasi tertulis yang diperhatikan:
\begin{itemize}
    \item Penggunaan bahasa formal dan ilmiah
    \item Struktur penulisan yang sistematis
    \item Penyajian data yang jelas dan informatif
    \item Penggunaan diagram, tabel, dan grafik yang efektif
    \item Sitasi dan referensi yang tepat
\end{itemize}

\vspace{0.5cm}

\textbf{B. Penyampaian Informasi K3L}

\textit{[Jelaskan bagaimana informasi K3L disampaikan secara efektif dalam laporan ini]}

\begin{itemize}
    \item Penggunaan visualisasi untuk memperjelas informasi
    \item Penyusunan SOP yang mudah dipahami
    \item Rekomendasi yang actionable dan spesifik
    \item \textit{[Tambahkan aspek lainnya]}
\end{itemize}

\vspace{0.5cm}

\subsubsection{Tanggung Jawab Akademik}

\textit{[Tunjukkan sikap bertanggung jawab dalam mengerjakan UTS]}

\textbf{A. Integritas Akademik}

\begin{itemize}
    \item Originalitas karya: \textit{[Jelaskan bahwa laporan adalah hasil kerja sendiri]}
    \item Penggunaan referensi: \textit{[Jelaskan semua sumber telah dikutip dengan benar]}
    \item Data observasi: \textit{[Jelaskan data diperoleh secara langsung dan valid]}
\end{itemize}

\vspace{0.5cm}

\textbf{B. Komitmen terhadap K3L}

\textit{[Jelaskan komitmen pribadi Anda terhadap penerapan K3L di masa depan]}

\begin{itemize}
    \item Kesadaran akan pentingnya K3L
    \item Kesiapan menerapkan ilmu K3L di dunia kerja
    \item Komitmen untuk terus belajar dan mengembangkan kompetensi K3L
\end{itemize}

\vspace{0.5cm}

\subsubsection{Evaluasi Diri dan Rencana Pengembangan}

\textit{[Lakukan evaluasi diri dan buat rencana pengembangan kompetensi]}

\textbf{Kekuatan yang Dimiliki:}
\begin{enumerate}
    \item \textit{[Sebutkan]}
    \item \textit{[Sebutkan]}
\end{enumerate}

\textbf{Area yang Perlu Ditingkatkan:}
\begin{enumerate}
    \item \textit{[Sebutkan]}
    \item \textit{[Sebutkan]}
\end{enumerate}

\textbf{Rencana Pengembangan:}
\begin{enumerate}
    \item \textit{[Rencana 1]}
    \item \textit{[Rencana 2]}
    \item \textit{[Rencana 3]}
\end{enumerate}

% ========== CATATAN ==========
% Pastikan bagian sikap akademik ditulis dengan:
% - Jujur dan reflektif
% - Menunjukkan pemahaman mendalam tentang K3L
% - Menggunakan bahasa first person untuk refleksi
% - Konkret dan spesifik, bukan generik