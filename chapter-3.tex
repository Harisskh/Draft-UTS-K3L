% ============================================
% CHAPTER 3: IDENTIFIKASI BAHAYA DAN ANALISIS RISIKO
% Bobot: CPMK5-6 = 25
% ============================================

% ========== BAGIAN A: IDENTIFIKASI JENIS BAHAYA ==========
\subsection{Identifikasi Jenis Bahaya}
\label{subsec:identifikasi-bahaya}

% Petunjuk:
% - Identifikasi berbagai jenis bahaya di tempat kerja
% - Klasifikasikan bahaya berdasarkan kategorinya
% - Berikan deskripsi detail untuk setiap bahaya

Berdasarkan observasi dan inspeksi di lokasi kerja, berikut identifikasi jenis-jenis bahaya yang ditemukan:

\subsubsection{Bahaya Fisik (Physical Hazards)}

\textit{[Identifikasi bahaya fisik yang ada di tempat kerja]}

Contoh bahaya fisik yang perlu diidentifikasi:
\begin{itemize}
    \item Kebisingan (noise)
    \item Getaran (vibration)
    \item Pencahayaan tidak memadai
    \item Suhu ekstrem (panas atau dingin)
    \item Radiasi (ionisasi atau non-ionisasi)
    \item Tekanan tidak normal
    \item \textit{[Tambahkan bahaya fisik lain yang ditemukan]}
\end{itemize}

\vspace{0.5cm}

\subsubsection{Bahaya Kimia (Chemical Hazards)}

\textit{[Identifikasi bahaya kimia yang ada di tempat kerja]}

Contoh bahaya kimia:
\begin{itemize}
    \item Bahan kimia beracun
    \item Gas berbahaya
    \item Debu dan partikel
    \item Uap dan aerosol
    \item Bahan mudah terbakar
    \item Bahan korosif
    \item \textit{[Tambahkan bahaya kimia lain yang ditemukan]}
\end{itemize}

\vspace{0.5cm}

\subsubsection{Bahaya Biologi (Biological Hazards)}

\textit{[Identifikasi bahaya biologi yang ada di tempat kerja]}

Contoh bahaya biologi:
\begin{itemize}
    \item Bakteri dan virus
    \item Jamur dan spora
    \item Parasit
    \item Hewan berbahaya
    \item Limbah medis/biologis
    \item \textit{[Tambahkan bahaya biologi lain yang ditemukan]}
\end{itemize}

\vspace{0.5cm}

\subsubsection{Bahaya Ergonomi (Ergonomic Hazards)}

\textit{[Identifikasi bahaya ergonomi yang ada di tempat kerja]}

Contoh bahaya ergonomi:
\begin{itemize}
    \item Postur kerja tidak ergonomis
    \item Gerakan berulang (repetitive motion)
    \item Pengangkatan manual beban berat
    \item Workstation tidak sesuai
    \item Durasi kerja berlebihan
    \item \textit{[Tambahkan bahaya ergonomi lain yang ditemukan]}
\end{itemize}

\vspace{0.5cm}

\subsubsection{Bahaya Psikososial (Psychosocial Hazards)}

\textit{[Identifikasi bahaya psikososial yang ada di tempat kerja]}

Contoh bahaya psikososial:
\begin{itemize}
    \item Stres kerja
    \item Beban kerja berlebihan
    \item Kekerasan di tempat kerja
    \item Shift kerja tidak teratur
    \item Kurangnya dukungan sosial
    \item \textit{[Tambahkan bahaya psikososial lain yang ditemukan]}
\end{itemize}

\vspace{0.5cm}

\subsubsection{Bahaya Mekanik (Mechanical Hazards)}

\textit{[Identifikasi bahaya mekanik yang ada di tempat kerja]}

Contoh bahaya mekanik:
\begin{itemize}
    \item Mesin bergerak tanpa pelindung
    \item Alat potong dan tajam
    \item Bagian berputar (rotating parts)
    \item Peralatan bertekanan
    \item Kendaraan dan alat berat
    \item \textit{[Tambahkan bahaya mekanik lain yang ditemukan]}
\end{itemize}

\vspace{0.5cm}

\subsubsection{Bahaya Listrik (Electrical Hazards)}

\textit{[Identifikasi bahaya listrik yang ada di tempat kerja]}

Contoh bahaya listrik:
\begin{itemize}
    \item Kontak langsung dengan arus listrik
    \item Instalasi listrik tidak standar
    \item Kabel rusak atau tidak terproteksi
    \item Grounding tidak memadai
    \item Overload listrik
    \item \textit{[Tambahkan bahaya listrik lain yang ditemukan]}
\end{itemize}

\vspace{1cm}

% ========== BAGIAN B: ANALISIS RISIKO ==========
\subsection{Analisis Risiko}
\label{subsec:analisis-risiko}

% Petunjuk:
% - Gunakan metode risk assessment (misalnya: Risk Matrix 5x5)
% - Tentukan nilai kemungkinan (likelihood) dan keparahan (severity)
% - Hitung tingkat risiko dan prioritaskan pengendalian

\subsubsection{Metode Penilaian Risiko}

Analisis risiko dilakukan menggunakan metode \textit{Risk Matrix 5x5} dengan kriteria sebagai berikut:

\textbf{Kriteria Kemungkinan (Likelihood):}
\begin{itemize}
    \item Nilai 1 (Sangat Jarang): Hampir tidak pernah terjadi (< 1 kali/10 tahun)
    \item Nilai 2 (Jarang): Pernah terjadi di industri sejenis (1 kali/5-10 tahun)
    \item Nilai 3 (Kadang-kadang): Dapat terjadi (1 kali/1-5 tahun)
    \item Nilai 4 (Sering): Sering terjadi (beberapa kali/tahun)
    \item Nilai 5 (Sangat Sering): Sangat sering terjadi (beberapa kali/bulan)
\end{itemize}

\textbf{Kriteria Keparahan (Severity):}
\begin{itemize}
    \item Nilai 1 (Dapat Diabaikan): Tidak ada cidera, kerugian minimal
    \item Nilai 2 (Minor): Cidera ringan, P3K, kerugian kecil
    \item Nilai 3 (Sedang): Cidera sedang, perlu perawatan medis, kerugian sedang
    \item Nilai 4 (Major): Cidera berat, cacat permanen, kerugian besar
    \item Nilai 5 (Catastrophic): Kematian, cacat total, kerugian sangat besar
\end{itemize}

\textbf{Tingkat Risiko:}
\begin{itemize}
    \item Risiko Rendah (1-4): Dapat diterima dengan monitoring
    \item Risiko Sedang (5-9): Perlu tindakan pengendalian
    \item Risiko Tinggi (10-15): Perlu tindakan segera
    \item Risiko Sangat Tinggi (16-25): Hentikan aktivitas sampai risiko terkendali
\end{itemize}

\vspace{0.5cm}

\subsubsection{Tabel Analisis Risiko}

\begin{landscape}
\begin{longtable}{|c|p{3cm}|p{3.5cm}|p{3.5cm}|c|c|c|p{4cm}|}
\caption{Analisis Risiko K3L} \label{tab:analisis-risiko} \\
\hline
\textbf{No} & \textbf{Jenis Bahaya} & \textbf{Sumber Bahaya} & \textbf{Potensi Akibat} & \textbf{Nilai Kemungkinan (L)} & \textbf{Nilai Keparahan (S)} & \textbf{Tingkat Risiko (LxS)} & \textbf{Rekomendasi Pengendalian} \\
\hline
\endfirsthead

\multicolumn{8}{c}%
{\tablename\ \thetable\ -- \textit{Lanjutan dari halaman sebelumnya}} \\
\hline
\textbf{No} & \textbf{Jenis Bahaya} & \textbf{Sumber Bahaya} & \textbf{Potensi Akibat} & \textbf{Nilai Kemungkinan (L)} & \textbf{Nilai Keparahan (S)} & \textbf{Tingkat Risiko (LxS)} & \textbf{Rekomendasi Pengendalian} \\
\hline
\endhead

\hline
\multicolumn{8}{r}{\textit{Bersambung ke halaman berikutnya}} \\
\endfoot

\hline
\endlastfoot

% Contoh baris pertama
1 & \textit{[Contoh: Kebisingan]} & \textit{[Mesin produksi]} & \textit{[Gangguan pendengaran]} & \textit{4} & \textit{3} & \textit{12 (Tinggi)} & \textit{[APD earplug/earmuff, enclosure mesin]} \\
\hline

% Baris untuk diisi
2 & & & & & & & \\
\hline
3 & & & & & & & \\
\hline
4 & & & & & & & \\
\hline
5 & & & & & & & \\
\hline
6 & & & & & & & \\
\hline
7 & & & & & & & \\
\hline
8 & & & & & & & \\
\hline
9 & & & & & & & \\
\hline
10 & & & & & & & \\
\hline

\end{longtable}
\end{landscape}

\vspace{0.5cm}

\subsubsection{Risk Matrix}

\begin{table}[h]
\centering
\begin{tabular}{|c|c|c|c|c|c|}
\hline
\multirow{2}{*}{\textbf{Keparahan (S)}} & \multicolumn{5}{c|}{\textbf{Kemungkinan (L)}} \\
\cline{2-6}
 & \textbf{1} & \textbf{2} & \textbf{3} & \textbf{4} & \textbf{5} \\
\hline
\textbf{5} & 5 & 10 & 15 & 20 & 25 \\
\hline
\textbf{4} & 4 & 8 & 12 & 16 & 20 \\
\hline
\textbf{3} & 3 & 6 & 9 & 12 & 15 \\
\hline
\textbf{2} & 2 & 4 & 6 & 8 & 10 \\
\hline
\textbf{1} & 1 & 2 & 3 & 4 & 5 \\
\hline
\end{tabular}
\caption{Risk Matrix 5x5}
\label{tab:risk-matrix}
\end{table}

\textbf{Keterangan Warna:}
\begin{itemize}
    \item Hijau (1-4): Risiko Rendah
    \item Kuning (5-9): Risiko Sedang
    \item Oranye (10-15): Risiko Tinggi
    \item Merah (16-25): Risiko Sangat Tinggi
\end{itemize}

\vspace{1cm}

% ========== BAGIAN C: ALAT BANTU IDENTIFIKASI BAHAYA ==========
\subsection{Alat Bantu Identifikasi Bahaya}
\label{subsec:alat-bantu}

% Petunjuk:
% - Jelaskan alat/metode yang digunakan untuk identifikasi bahaya
% - Diskusikan kelebihan dan kekurangan masing-masing alat
% - Berikan contoh penerapan

\subsubsection{Metode Identifikasi Bahaya}

Dalam melakukan identifikasi bahaya dan analisis risiko, digunakan beberapa metode dan alat bantu:

\textbf{1. Job Safety Analysis (JSA)}

\textit{[Jelaskan penggunaan JSA dalam mengidentifikasi bahaya pada setiap langkah pekerjaan]}

Kelebihan:
\begin{itemize}
    \item Analisis detail per langkah kerja
    \item Mudah dipahami pekerja
    \item Dapat digunakan sebagai bahan pelatihan
\end{itemize}

\textbf{2. Hazard and Operability Study (HAZOP)}

\textit{[Jelaskan jika menggunakan metode HAZOP, terutama untuk proses industri]}

\textbf{3. What-If Analysis}

\textit{[Jelaskan penggunaan analisis What-If untuk brainstorming bahaya potensial]}

\textbf{4. Checklist Inspeksi K3L}

\textit{[Jelaskan penggunaan checklist untuk inspeksi rutin]}

Contoh checklist yang digunakan:
\begin{itemize}
    \item Checklist area kerja
    \item Checklist peralatan dan mesin
    \item Checklist APD
    \item Checklist housekeeping
\end{itemize}

\textbf{5. Safety Walk-Through}

\textit{[Jelaskan metode observasi langsung di lapangan]}

\textbf{6. Failure Mode and Effect Analysis (FMEA)}

\textit{[Jelaskan jika menggunakan FMEA untuk analisis kegagalan sistem]}

\vspace{0.5cm}

\subsubsection{Tools dan Teknologi Pendukung}

\textit{[Jelaskan alat bantu teknologi yang digunakan]}

\begin{itemize}
    \item Software HIRA (Hazard Identification and Risk Assessment)
    \item Alat ukur lingkungan kerja (sound level meter, lux meter, dll.)
    \item Aplikasi mobile untuk pelaporan bahaya
    \item Database bahaya dan risiko
    \item \textit{[Tambahkan tools lain yang digunakan]}
\end{itemize}

\vspace{0.5cm}

\subsubsection{Dokumentasi dan Pelaporan}

\textit{[Jelaskan sistem dokumentasi identifikasi bahaya]}

\begin{itemize}
    \item Format laporan identifikasi bahaya
    \item Sistem penyimpanan dan updating data
    \item Komunikasi hasil identifikasi kepada stakeholder
    \item Review dan update berkala
\end{itemize}

% ========== CATATAN ==========
% Pastikan:
% - Tabel analisis risiko diisi dengan data observasi nyata
% - Perhitungan tingkat risiko akurat
% - Rekomendasi pengendalian mengikuti hirarki kontrol
% - Semua bagian didukung dengan data dan referensi